\section{Conclusion}\label{sec:conclusion}

Our system demonstrates the feasibility of securing Web resources in a
corporate environment based on the physical location of users as
implied by a building access control system.  The system uses dynamic
group membership in a directory service such as Active Directory to
store the last known physical location of users.  It updates their
locations as they swipe their access cards on a RFID reader.  All
events are logged to a SIEM service, and, in the case of an invalid
access attempt, the threat actor is directed to a honeypot that flags
the event as high priority.  The system can prevent both internal and
external threats without placing additional burden on users.  It is
constructed from open-source and proprietary off-the-shelf components
in a vendor-neutral manner.  It adds a additional layer of defence
when securing Web resources.

Our approach follows the principles of zero trust architecture (ZTA)
as outlined in Sect.~\ref{sec:related-work}.  The physical location of
a user determines the appropriate defence around a Web resource.
Additionally, dynamic group membership is an example of role-based
access control (RBAC), and specifically, dynamic role-based access
control (DRBAC) (see Sect.~\ref{sec:related-work}): a user's role is
automatically adjusted based on context.  The real-time evaluation of
the context is performed by the directory service.  Our system is a
proof-of-concept but it can be extended in many ways.  For example, we
could incorporate temporal information from calendars and attendance
trackers, e.g., holidays, business travel, medical leave, etc.  We
could use fine-grained location information~\cite{kriplean-et-al-07}
and include geospatial data from mobile devices and laptops.

