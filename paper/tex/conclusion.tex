\section{Conclusion}\label{sec:conclusion}

Our system demonstrates the feasibility of securing Web resources in a
corporate environment based on the physical location of users as
implied by a building access control system.  The system uses dynamic
group membership in a directory service such as Active Directory to
store the last known physical location of users.  It updates their
locations as they swipe their access cards on a RFID reader.  All
events are logged to a SIEM service, and, in the case of an invalid
access attempt, the threat actor is directed to a honeypot that flags
the event as high priority.  The system can prevent both internal and
external threats without placing additional burden on users.  It is
constructed from open-source and proprietary off-the-shelf components
in a vendor-neutral manner.  It adds an additional layer of defence
when securing Web resources.

Our approach follows the principles of zero trust architecture (ZTA)
as outlined in Sect.~\ref{sec:related-work}.  The physical location of
a user determines the appropriate defence around a Web resource.
Additionally, dynamic group membership is an example of role-based
access control (RBAC), and specifically, dynamic role-based access
control (DRBAC) (see Sect.~\ref{sec:related-work}): a user's role is
automatically adjusted based on context.  The real-time evaluation of
the context is performed by the directory service.  Our system is a
proof-of-concept but it can be extended in many ways.  For example, we
could incorporate temporal information from calendars and attendance
trackers, e.g., holidays, business travel, medical leave, etc.  We
could use fine-grained location information~\cite{kriplean-et-al-07}
and include geospatial data from mobile devices and laptops.

Our system has limitations.  Firstly, RFID tags and readers are
vulnerable to communication layer weaknesses, man-in-the-middle
attacks, and physical attacks, amongst
others~\cite{ranasinghe-cole-06}.  Secondly, location-based profiling
can lead to a culture of tracking and surveillance and violate the
privacy of employees~\cite{ostojic-et-al-07}.  Thirdly, the mapping of
connection sources to physical locations is imperfect due to
technologies such as proxies and VPNs~\cite{luna-21}.  This leads to
false positives (employees flagged as threats) and false negatives
(threats going undetected).  Finally, our system requires additional
configuration and management, e.g. the locations of the card readers
must be paired with expected connection sources and the SIEM system
needs regular monitoring.  There can be challenges in upgrading legacy
infrastructure and integrating the components as described above.
Furthermore, the integration can impact application performance, e.g.,
the additional overhead on each access attempt.

In our future work, we would like to evaluate the system using a
mid-size, real-world deployment.  The limitations above can only be
addressed in the wild.  For example, we could use progressive
deployment strategies to evaluate the trade-off between the additional
configuration and management and the rate of threat detection, e.g.,
by performing a split test between two buildings in the same corporate
environment.
