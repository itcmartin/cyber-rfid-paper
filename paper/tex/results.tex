\section{Results}\label{sec:results}

We configured the system as described in the previous section.  The
goal of the study was to show the feasibility of the integration
between the various components, and to demonstrate that dynamic group
membership based on real-world location can secure Web resources.  We
performed two tests (Sect.~\ref{sec:case-one} and
Sect.~\ref{sec:case-two}) to demonstrate this aspect of the system.

\subsection{Internal User with External Threat}\label{sec:case-one}

In the first case the user, John Doe, enters their office during
normal working hours and swipes their card at the door using an RFID
tag.  Their group membership is set to Internal and user can then
access the resource internally.  Meanwhile an external threat actor
attempts to login from outside the office while the user is at work.
They are denied access and they have their username and IP address
logged to the SIEM service (WUG) as a breach attempt.  This requires
no extra overhead on the user to secure their credentials.  The
timeline of events is as follows:

\begin{figure*}
  \centerline{\includegraphics[width=\textwidth]{img/whatsup-gold-alert}}
  \caption{...}\label{fig:whatsup-gold-alert}
\end{figure*}

\begin{enumerate}
\item John Doe enters their office and swipes their access card.
\item The user's group membership is changed from External to
  Internal.  This event is logged to the SIEM service from the
  Raspberry Pi.
\item When the user gets to their desk they access the Web resource
  internally.
\item The DNS points to the LM for DNS resolution of the Web resource
  and since it is an internal request the user is sent to the internal
  service.
\item The ESP requires the user to login.
\item The user's group membership is checked by the ESP SSO system and
  they are connected to the appropriate Web resource.
\item An external threat actor attempts to access the Web resource
  using John Doe's credentials.
\item The DNS points the threat actor to the LM for DNS resolution.
\item Using the correct credentials, the external threat actor logs in
  via the ESP SSO page.
\item The group membership is read as Internal, but the source is
  external, so the threat actor is directed to a ``server
  unavailable'' page and their username and source IP address are
  logged to the SIEM service (see Fig.~\ref{fig:whatsup-gold-alert}).
\end{enumerate}

\subsection{External User with Internal Threat}\label{sec:case-two}
