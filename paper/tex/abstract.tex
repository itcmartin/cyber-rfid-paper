It is a fine balance to provide employees with approved and convenient
access to corporate systems, while preventing all forms of undesired
access.  In the case of Web resources, we can better strike this
balance by cross-checking the source of the connecting device with an
employee's physical location as reported by a building's access
control system.  To this end, we assemble a collection of open-source
and proprietary components, including an RFID card reader, a directory
service with dynamic group membership, a SIEM system, and a honeypot.
We show that this assemblage can protect against internal and external
threats without imposing additional burden on employees.

Many corporate environments already have the requisite components, in
particular, a building access control system and a centralised
directory service such as Microsoft Active Directory for
authentication and authorisation.  We configure the components in a
loosely coupled architecture that utilises dynamic groups in the
directory service to store the current location of employees based on
their swipe access records.  This novel use of dynamic groups can be
combined with existing group-based rules.  We proxy all requests to
the Web resources through a load balancer that consults the directory
service and cross-checks the connection sources with the stored
location of the employees.  This defends against compromised
credential attacks from external and internal threat actors: we
illustrate this using two case studies.  The components are
interchangeable and the architecture can be adapted to many
environments.  Our approach follows the principles of the zero trust
security model and dynamic role-based access control.
