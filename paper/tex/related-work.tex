\section{Related Work}\label{sec:related-work}

We categorise related work into four areas: using RFID technology to
integrate physical and digital security; multi-factor authentication,
specifically schemes that utilise location as one of the factors;
dynamic role-based access control; and the zero trust security model
or zero trust architecture.

RFID can enhance and modernise conventional approaches to access
control and authentication.  Clarke~\cite{clarke-11} surveys a range
of transparent user authentication schemes, including schemes that
rely on RFID tags and other contactless tokens.  Farooq el
al.~\cite{farooq-et-al-14}, Larchikov et al.~\cite{larchikov-et-al-14}
and Woo-Garcia et al.~\cite{woo-garcia-et-al-16} describe access
control systems that employ RFID tags to differentiate between valid
and invalid users.  All of the systems read the RFID tags at the
entrances and exits of a building.  Kriplean et
al.~\cite{kriplean-et-al-07} deploy a building-wide RFID
infrastructure with eighty RFID readers that gather fine-grained
location information.  They consider ways of creating utility while
respecting the privacy of users.  Ostoji\'c et
al.~\cite{ostojic-et-al-07} deploy a similar system to manage access
to a parking lot.  Our system uses a Raspberry Pi connected to an RFID
door entry system in a similar manner.  There are many concerns
surrounding RFID tags including cloning, man-in-the-middle attacks,
denial-of-service attacks, communication layer weaknesses, and
physical attacks (see, e.g., Ranasinghe and
Cole~\cite{ranasinghe-cole-06}).

Ometov et al.~\cite{ometov-et-al-18} surveys multi-factor
authentication (MFA) schemes: they consider various types of MFA
sensors including geolocation sensors.  Location-based MFA schemes,
such as the one described by Ramatsakane and
Leung~\cite{ramatsakane-leung-17}, seek to balance usability and
security.  Suo et al.~\cite{suo-et-al-22} use \textit{location
  signatures} to secure automated vehicles.  A location signature is a
geo- and time-stamped message issued by a trusted device that attests
to a vehicle's presence in a particular location at a given
time~\cite{chen-et-al-09}.  We use the location of a user's access
card as an authentication factor.

Dynamic role-based access control (DRBAC) is an extension of
traditional role-based access control
(RBAC)~\cite{franqueira-wieringa-12} that enables the automatic
adjustment of user roles based on factors like context, behaviour, and
risk assessment.  Unlike static RBAC, DRBAC assigns roles dynamically
in response to real-time conditions, ensuring users have an
appropriate access level.  This enhances security and adaptability but
requires real-time evaluation that can be more complex to implement.
Uzun et al.~\cite{uzun-et-al-12} extend the traditional RBAC model to
handle temporal and geospatial constraints.  Luo et
al.~\cite{luo-et-al-16} extend the RBAC model to a cloud environment
where roles are determined by the security state and network
availability of the resources.  Chatterjee et
al.~\cite{chatterjee-et-al-20} describe a decentralised RBAC model
that relies on a blockchain with smart contracts.  They implement a
proof-of-concept on the Ethereum virtual machine (EVM) and quantify
its computational cost in terms of EVM gas.  Finally, Liu et
al.~\cite{liu-et-al-18} survey insider threats and describe systems
where host, network and contextual data can identify such threats.
Our system has a related capability: it can flag insider threats by
comparing the physical location of the target with the source of the
connection.

The zero trust security model or zero trust architecture (ZTA) shifts
cybersecurity defences from static, network-based perimeters to users,
assets, and resources that are dynamic and
perimeter-less~\cite{syed-et-al-22}.  Rose et al.~\cite{rose-et-al-20}
and Garbis and Chapman~\cite{garbis-chapman-21} define ZTA and
describe its logical components.  Bertino~\cite{bertino-21} highlights
management and deployment as the main challenges of ZTA.\@ Ross et
al~\cite{ross-et-al-21} show that multiple cyber resiliency
techniques, can be integrated into the design and deployment of ZTA.\@
Yao et al.~\cite{yao-et-al-21} combine ZTA and trust-based access
control (TBAC) to evaluate the trust of users and compare those
evaluations against trust thresholds.  In the same vein as above, Meng
et al.~\cite{meng-et-al-22} utilise a blockchain to decentralise the
operation of the trusted nodes in a ZTA.\@ Identifying the location of
a user and changing their group membership based on that location,
follows the principles of ZTA.\@ By making the process fully automated
and transparent, we can minimise the management and deployment
overhead.
