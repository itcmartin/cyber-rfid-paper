\section{Related Work}\label{sec:related-work}

We categorise related work into four areas: using RFID technology to
integrate physical and digital security; multi-factor authentication,
specifically schemes that utilise location as one factor; dynamic
role-based access control; and the zero trust security model or zero
trust architecture.

RFID can enhance and modernise conventional approaches to access
control and authentication.  Clarke~\cite{clarke-11} surveys a range
of transparent user authentication schemes, including schemes that
rely on RFID tags and other contactless tokens.  Farooq el
al.~\cite{farooq-et-al-14}, Larchikov et al.~\cite{larchikov-et-al-14}
and Woo-Garcia et al.~\cite{woo-garcia-et-al-16} describe access
control systems that employ RFID tags to differentiate between valid
and invalid users.  All of the systems read the RFID tags at the
entrances and exits of a building.  Kriplean et
al.~\cite{kriplean-et-al-07} deploy a building-wide RFID
infrastructure with eighty RFID readers that gather fine-grained
location information.  They consider ways of creating utility while
respecting the privacy of users.  Ostoji\'c et
al.~\cite{ostojic-et-al-07} deploy a similar system to manage access
to a parking lot.  There are many concerns surrounding RFID tags
including cloning, man-in-the-middle attacks, denial-of-service
attacks, communication layer weaknesses, and physical attacks (see,
e.g., Ranasinghe and Cole~\cite{ranasinghe-cole-06}).

Ometov et al.~\cite{ometov-et-al-18} surveys multi-factor
authentication (MFA) schemes: they consider various types of MFA
sensors including geolocation sensors.  Location-based MFA schemes,
such as the one described by Ramatsakane and
Leung~\cite{ramatsakane-leung-17} seeks to balance usability and
security.  Suo et al.~\cite{suo-et-al-22} use \textit{location
  signatures} to secure automated vehicles.  A location signature is a
geo- and time-stamped message issued by a trusted device that attests
to a vehicle's presence in a particular location at a given
time~\cite{chen-et-al-09}.

%% dynamic role-based access control
%% \cite{uzun-et-al-12}
%% \cite{luo-et-al-16}
%% \cite{liu-et-al-18} (insider threats)
%% \cite{chatterjee-et-al-20}

%% zero trust architecture (ZTA)
%% \cite{rose-et-al-20}
%% \cite{bertino-21}
%% \cite{garbis-chapman-21}
%% \cite{ross-et-al-21}
%% \cite{yao-et-al-21}
%% \cite{meng-et-al-22}
%% \cite{syed-et-al-22}
