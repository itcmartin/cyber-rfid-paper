\section{Introduction}\label{sec:introduction}

To secure a Web resource in a corporate environment, there are two
modes of access to consider: an on-site user internally accessing the
resource and a remote user externally accessing the same resource.  In
the first case, the user is physically located in a building that may
be guarded by an access control system.  In the second case, the user
is located outside of those buildings.  However, attempts to access
the Web resource are often handled similarly in both cases: the
physical location of the user is ignored.  In this paper we describe a
system that combines location with credentials when securing Web
resources.  Specifically, it utilises RFID tags and readers, dynamic
groups in Microsoft Active Directory or a similar directory service,
SIEM alerting, and honeypots.  The system provides an additional layer
of defence that imposes no extra burden on users.

\begin{quote}
  \em{There is no silver bullet solution with cybersecurity; a layered
    defence is the only viable defence.}

  \attrib{James Scott, ICIT, https://www.icitech.org/}
\end{quote}

We deployed the system using a combination of open-source and
proprietary off-the-shelf components.  The components are loosely
coupled and interchangeable.  The approach can be adapted to a variety
of corporate environments that utilise access control systems in their
buildings.  Furthermore, the system can be layered on top of existing
infrastructure.

The system distinguishes between internal and external threats and
records additional context during failed attempts to access a Web
resource.  Insider threats can be difficult and time-consuming to
detect~\cite{liu-et-al-18}.  Our system flags occasions where a
resource is accessed internally using credentials belonging to a user
that is currently operating remotely.  This runs counter to many
existing systems where internal traffic is assumed to be safer than
external traffic.

This paper is organised as follows.  In Sect.~\ref{sec:related-work}
we review related work, including RFID technology, multi-factor
authentication, dynamic role-based access control and the zero trust
security model.  In Sect.~\ref{sec:method} we describe our system:,
the high-level architecture and the various components and their
configuration.  We detail our results in Sect.~\ref{sec:results}.
Specifically, we consider two use cases: an internal user with an
external threat, and an external user with an internal threat.  In
both cases, we demonstrate the system's ability to flag threats.
Finally, we conclude in Sect.~\ref{sec:conclusion}.
